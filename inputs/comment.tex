%%%%%%%%%%%%%%%%%%%%%%%%%%%%%%%%%%%%%%%%%%%%
% Comment.sty   version 3.0, 3 September 1992
% selectively in/exclude pieces of text: the user can define new
% comment versions, and each is controlled separately.
% Special comments can be defined where the user specifies the
% action that is to be taken with each comment line.
%
% This style can be used with plain TeX or LaTeX, and probably
% most other packages too.
%
% Author
%    Victor Eijkhout
%    Department of Computer Science
%    University Tennessee at Knoxville
%    104 Ayres Hall
%    Knoxville, TN 37996
%    USA
%
%    eijkhout@cs.utk.edu
%
% Usage: all text included in between 
%    \comment ... \endcomment
% or \begin{comment} ... \end{comment}
% is discarded. The closing command should appear on a line
% of its own. No starting spaces, nothing after it.
% This environment should work with arbitrary amounts
% of comment. 
%
% Other 'comment' environments are defined by
% and are selected/deselected with
% \includecomment{versiona}
% \excludecoment{versionb}
%
% These environments are used as
% \versiona ... \endversiona
% or \begin{versiona} ... \end{versiona}
% with the closing command again on a line of its own.
%
% Special comments are defined as
% \specialcomment{name}{before commands}{after commands}
% where the second and third arguments are executed before
% and after each comment. By defining a control sequence 
% \Thiscomment#1{...} in the before commands the user can
% specify what is to be done which each comment line.
%
% Basic approach: 
% to comment something out, scoop up  every line in verbatim mode
% as macro argument, then throw it away.
% For inclusions, both the opening and closing comands 
% are defined as noop
%
% This file copyright Victor Eijkhout. All nonprofit use granted

\def\makeinnocent#1{\catcode`#1=12 }
\def\csarg#1#2{\expandafter#1\csname#2\endcsname}

\def\TreatAsComment#1{\begingroup
    \def\CurrentComment{#1}%
    \let\do\makeinnocent \dospecials 
    \makeinnocent\^^L% and whatever other special cases
    \endlinechar`\^^M \catcode`\^^M=12 \xComment}
{\catcode`\^^M=12 \endlinechar=-1 %
 \gdef\xComment#1^^M{\def\test{#1}
      \csarg\ifx{PlainEnd\CurrentComment Test}\test
          \def\next{\endgroup\AfterComment}%
      \else \csarg\ifx{LolliEnd\CurrentComment Test}\test
          \def\next{\endgroup\AfterComment}%
      \else \csarg\ifx{LaLaEnd\CurrentComment Test}\test
            \edef\next{\endgroup\noexpand\AfterComment
                       \noexpand\end{\CurrentComment}}
      \else \ThisComment{#1}\let\next\xComment
      \fi \fi \fi \next}
}

\def\includecomment
 #1{\message{Including comment '#1'}%
    \expandafter\def\csname#1\endcsname{}%
    \expandafter\def\csname end#1\endcsname{}}
\def\excludecomment
 #1{\message{Excluding comment '#1'}%
    \csarg\def{#1}{\let\AfterComment\relax
           \def\ThisComment####1{}\TreatAsComment{#1}}%
    {\escapechar=-1\relax
     \csarg\xdef{PlainEnd#1Test}{\string\\end#1}%
     \csarg\xdef{LolliEnd#1Test}{\string\\#1Stop}%
     \csarg\xdef{LaLaEnd#1Test}{\string\\end\string\{#1\string\}}%
    }}
\long\def\specialcomment
 #1#2#3{\message{Special comment '#1'}%
    \csarg\def{#1}{\def\ThisComment{}\def\AfterComment{#3}#2%
           \TreatAsComment{#1}}%
    {\escapechar=-1\relax
     \csarg\xdef{PlainEnd#1Test}{\string\\end#1}%
     \csarg\xdef{LolliEnd#1Test}{\string\\#1Stop}%
     \csarg\xdef{LaLaEnd#1Test}{\string\\end\string\{#1\string\}}%
    }}
\excludecomment{comment}

\endinput

% a test of the special comments
\newcount\comlines
\specialcomment{countedcomment}
 {\comlines=-1\relax %note that the rest of the first line is also comment
  \def\ThisComment{\global\advance\comlines1\relax}}
 {\endgraf **Comment: \number\comlines\ line(s) removed**\endgraf}

This is a line of text
\countedcomment
Oneline
\endcountedcomment
another line of text
\countedcomment
One line
Two line
Three line
\endcountedcomment
last line of text
\end
