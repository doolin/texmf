%%%% Various math commands.
%  The following block of commands must be used in math mode only.

%  Miscellaneous program names.
\def\matlab{{\tt Matlab}}
\def\maple{{\tt Maple}}

%  I get tired of doing the following operation:
\newcommand{\mand}{%
\quad\mbox{and}\quad%
}

%  This command (\sub) will be superseded by the \units command.
\newcommand{\sub}[1]{%
_{\rm #1} }

\newcommand{\units}[1]{%
_{\rm #1} }

\newcommand{\s}[1]{%
_{\rm  \tiny #1} }

\renewcommand{\gg}{\gamma}
\newcommand{\gr}{\rho}
\newcommand{\f}{\frac}


\newcommand{\ee}[1]{%
\times 10^{#1} }

%%%%%%%%%% Math environment macros

\newcommand{\nn}{\nonumber}

\newcommand{\beq}{\begin{equation}}
\newcommand{\eeq}{\end{equation}}

\newcommand{\beqa}{\begin{eqnarray}}
\newcommand{\eeqa}{\end{eqnarray}}

\newcommand{\colthreevec}[3]{%
\left\{\begin{array}{c}#1\\#2\\#3
\end{array}\right\}}

\newcommand{\colfourvec}[4]{%
\left\{\begin{array}{c}#1\\#2\\#3\\#4
\end{array}\right\}}

\newcommand{\mattwobytwo}[4]{%
\left[\begin{array}{cc}#1 & #2\\#3 & #4
\end{array}\right]}


\newcommand{\matthreebythree}[9]{%
\left[\begin{array}{ccc}#1 & #2 & #3\\%
                        #4 & #5 & #6\\%
                        #7 & #8 & #9 
\end{array}\right]}

%%%%%%%%%%%%%%%%%%%%%%%%%%%%%%%%%%%%%%%%%%%%%%%%%%%%%%%


%  This is a macro for gapping a subfigure:
\def\goodgap{% 
\hspace{\subfigtopskip}% 
\hspace{\subfigbottomskip}}


%%%%%%%%%%%%%%%%%%%%%%%%%%%%%%%%%%%%%%%%%%%%%%%%%%%%%

% Here is Darcy's Law.  Must be in math mode.

\newcommand{\Darcy}{%
Q_i = AK_{ij}\frac{\partial h}{\partial x_i}
}

%%%%%%%%%%%%%%%%%%%%%%%%%%%%%%%%%%%%%%%%%%%%%%%%%%%

%Here is a macro for units in engineering work;
% Must be in math mode.

\newcommand{\unit}[1]{%
{\rm \, #1}
}


%%%%%%%%%%%%%%%%%%%%%%%%%%%%%%%%%%%%%%%%%%%%%%

%  An orientation tensor macro.
%  must use in math mode for now.  Jane Hahns book
%  shows how to get around that limitation.

\newcommand{\otensor}[1]{%
#1= \frac{1}{N}\left[\begin{array}{lll}
\sum l_i^2 & \sum l_im_i & \sum l_in_i\\
\sum m_il_i & \sum m_i^2 & \sum m_in_i\\
\sum n_il_i & \sum n_im_i & \sum n_i^2
\end{array}\right]
}

%%%%%%%%%%%%%%%%%%%%%%%%%%%%%%%%%%%%%%%%%%%%%%
% Some differential macros.

\newcommand{\dqdx}{\frac{\partial q}{\partial x}}
\newcommand{\dNdx}{\frac{d\{N_k\}}{dx}}
\newcommand{\dNTdx}{\frac{d\{N_k\}^T }{dx}}
\newcommand{\dTdx}{\frac{\partial T}{\partial x}}
\newcommand{\dqdt}{\frac{\partial q}{\partial t}}
\newcommand{\ddx}{\frac{\partial }{\partial x}}
\newcommand{\ddr}{\frac{\partial }{\partial r}}
\newcommand{\dqdn}{\frac{\partial q }{\partial n}}
\newcommand{\dphidn}{\frac{\partial \phi }{\partial n}}
\newcommand{\dQdt}{\frac{\partial Q }{\partial t}}
\newcommand{\dQdr}{\frac{\partial Q }{\partial r}}

