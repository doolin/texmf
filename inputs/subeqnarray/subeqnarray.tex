%%
%% This is file `subeqnarray.tex',
%% generated with the docstrip utility.
%%
%% The original source files were:
%%
%% subeqnarray.dtx  (with options: `sample')
%% 
%% This file is part of the subeqnarray package.
%% ---------------------------------------------
%% 
%% Copyright (C) 1988--1999 Johannes Braams. All rights reserved.
%% 
%% This program can be redistributed and/or modified under the terms
%% of the LaTeX Project Public License Distributed from CTAN
%% archives in directory macros/latex/base/lppl.txt; either
%% version 1 of the License, or any later version.
%% 
%% Run this file through LaTeX to demonstrate the features
%% of the subeqnarray package.
%% 
%% Copyright (C) 1988--1999 by Johannes Braams, <JLBraams@cistron.nl>
%%                          all rights resserved
%%
%% This program can be redistributed and/or modified under the terms
%% of the LaTeX Project Public License Distributed from CTAN
%% archives in directory macros/latex/base/lppl.txt; either
%% version 1 of the License, or any later version.
%%
%% Error reports please to: J. Braams
%%                          TeXniek
%%                          Kooienswater 62
%%                          2715 AJ Zoetermeer
%%                          The Netherlands
%%                  Email:  <JLBraams@cistron.nl>
\ProvidesFile{subeqnarray.tex}
              [1999/03/03 v2.1b subeqnarray package]
\documentclass[fleqn]{article}
\usepackage{subeqnarray}
\begin{document}
This document shows an example of the use of the \emph{subeqnarray}
environment. Here is one:
\begin{subeqnarray}
\label{eqw}
\slabel{eq0}
 x & = & a \times b \\
\slabel{eq1}
 & = & z + t\\
\slabel{eq2}
 & = & z + t
\end{subeqnarray}
The first equation is number~\ref{eq0}, the last is~\ref{eq2}. The
equation as a whole can be referred to as equation~\ref{eqw}.

To show that equation numbers behave normally, here's an
\emph{eqnarray} environment.
\begin{eqnarray}
\label{eq10}
 x & = & a \times b \\
\label{eq11}
 & = & z + t\\
\label{eq12}
 & = & z + t
\end{eqnarray}

These are equations~\ref{eq10},~\ref{eq11} and~\ref{eq12}.
\end{document}
\endinput
%%
%% End of file `subeqnarray.tex'.
